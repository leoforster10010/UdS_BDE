\documentclass{article}
\usepackage{iftex}
\ifluatex
	\usepackage{fontspec}
\else
	\usepackage[utf8]{inputenc} %Eingabekodierung nach UTF8-Unicode, Erlaubt direkte Eingabe von Umlauten etc.
\fi
\usepackage{amssymb}
\usepackage{amsmath}
\usepackage[ngerman]{babel}
\usepackage{csquotes}

\usepackage[headheight=24pt,heightrounded]{geometry}
\usepackage{graphicx}
\usepackage{titlesec}
\usepackage{fancyhdr}
\usepackage{enumerate}
\usepackage{listings}

\lstdefinestyle{mystyle}{
	keywordstyle=\color{magenta},
	basicstyle=\ttfamily\footnotesize,
	breakatwhitespace=false,         
	breaklines=true,                 
	captionpos=b,                    
	keepspaces=true,                 
	numbers=left,                    
	numbersep=5pt,                  
	showspaces=false,                
	showstringspaces=false,
	showtabs=false,                  
	tabsize=2
}
\lstset{style=mystyle}


\usepackage{xcolor}
\definecolor{urlcolor}{RGB}{0,136,204}
\definecolor{linkcolor}{RGB}{204,0,34}
\usepackage[colorlinks,urlcolor=urlcolor,linkcolor=linkcolor]{hyperref}

% TODOs
\usepackage[colorinlistoftodos,prependcaption,textsize=tiny]{todonotes}

% coloneqq
\usepackage{mathtools}

\usepackage{qtree}

% Blank lines instead of indented first line
\setlength{\parindent}{0pt}
\setlength{\parskip}{\baselineskip}

\newcommand{\sheet}[1]{
\pagestyle{fancy}
\fancyhead[L]{Big Data Engineering  \\ Assignment 9}
\fancyhead[R]{Leo Forster \\ Maximilian Prinz \\ Bastian Simon}

\titleformat{\section}
{\normalfont\Large\bfseries}{\thesection.}{1em}{}

}

% Hier kommen Vorlesungs- und Dokumentspezifische Makros

\newcommand{\NN}{\mathbb{N}}
\newcommand{\ZZ}{\mathbb{Z}}
\newcommand{\QQ}{\mathbb{Q}}
\newcommand{\RR}{\mathbb{R}}
\newcommand{\CC}{\mathbb{C}}
\newcommand{\inv}{^{-1}}
\newcommand{\lpar}{\left(}
\newcommand{\rpar}{\right)}

\newcommand{\limn}{{\lim_{n\to\infty}}}
\sheet{1}
\begin{document}

\section{ Relational Algebra and SQL}	

\subsection{a$)$}

\begin{lstlisting}[language=SQL]
	-- 1.
	SELECT DISTINCT b.title
	FROM books AS b
	JOIN authors AS a ON b.author = a.aid
	WHERE b.year > 1989 AND a.salary != 1500;
\end{lstlisting}

\begin{lstlisting}[language=SQL]
	-- 2.
	SELECT b.genre
	FROM (
		SELECT *
		FROM borrow b
		JOIN books ON b.book = books.bid
		WHERE b.borrow_date = '16.05.2024'
		GROUP BY b.genre
		HAVING COUNT(*) > 10
	)
\end{lstlisting}

\subsection{b$)$}

\begin{itemize}
	\item[1.] $\pi * ({\sigma}_{late\_fees = 0} members) ({\sigma}_{city \neq 'Saarbruecken'} libraries) \Join ({\sigma}_{member = mid} membership)$
	\item[2.] ${\pi}_{avg\_late\_fees} ()$
\end{itemize}
	
\section{Natural Language and SQL}	

\subsection{a$)$}

\begin{lstlisting}[language=SQL]
	-- 1.
	SELECT DISTINCT b.car
	FROM buy b
	LEFT JOIN repair r ON b.car = r.car AND r.reason = 'broken clutch'
	WHERE YEAR(b.date_of_purchase) < 2015 AND r.car IS NULL;
\end{lstlisting}

\begin{lstlisting}[language=SQL]
	-- 2.
	SELECT m.mid 
	FROM mechanics m
	JOIN repair r ON m.mid = r.mechanic
	JOIN cars c ON r.car = c.id
	GROUP BY m.mid
	HAVING COUNT(r.start) > 5 AND MAX(DATEDIFF(MONTH, r.start, r.end)) < 6 AND c.horsepower = 250;  
	
\end{lstlisting}

\subsection{b$)$}

\begin{itemize}
	\item[1.] Find the unique cities and salaries of mechanics who have serviced more than one Volkswagen Golf GTI vehicle
	\item[2.] Give a list of full names of the top ten earning mechanics, sorted by decreasing age, who earn more than 4000 per year and don’t have the last name Smith.
\end{itemize}
	
\section{SQL Debugging}

\subsection{}

\begin{itemize}
	\item GROUP BY should be after the WHERE
	\item JOIN should use 'c.id' and 'p.cameraId'
	\item NULL AS NOT\_NULL doesn't provide any meaningful information for this query
	\item 'location' is an ambigouus colum definition. Should be 'p.location'.
\end{itemize}

\begin{lstlisting}[language=SQL]
	-- Corrected Query
	SELECT p.location
	FROM cameras AS c 
	JOIN photos AS p ON c.id = p.cameraId
	WHERE location LIKE '%bruecken'
	GROUP BY location;
\end{lstlisting}

\subsection{}

\begin{itemize}
	\item 'measureOfAge' is an alias for a aggregated column, so it can't be used in the GROUP BY clause.
	\item JOIN on employees is needed for 'experience' in the WHERE clause.
	\item sum() on 's.numGreyHairs' is unnecessary, since it will only sum up single values.
\end{itemize}

\begin{lstlisting}[language=SQL]
	-- Corrected Query
	SELECT s.numGreyHairs AS measureOfAge
	FROM seniors AS s
	JOIN employees AS e ON s.employeeId = e.personId
	WHERE e.experience > 42;
\end{lstlisting}

\subsection{}

\begin{itemize}
	\item HAVING is used on GROUP BY aggregations, should be WHERE instead.
	\item Without aggregation sum() on 'experience' is unnecessary, since it will only sum up single values.
\end{itemize}

\begin{lstlisting}[language=SQL]
	-- Corrected Query
	SELECT salary
	FROM employees
	WHERE experience > 5;
\end{lstlisting}

\end{document}